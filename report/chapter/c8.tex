%!TEX root = ../document.tex
\chapter{Project Change Management}

\section{Raise the change}

To make a change, you must first fill out the REQUEST FOR CHANGE (RFC). The manager handed over to the other project manager. The recipient project manager will evaluate the RFC's technical reliability and impact on the entire project. The RFC agreed by the recipient project manager will submit the project leadership team approval, and the unapproved RFC will be returned to the applicant project manager. Any disputes that cannot be resolved by the project manager of both parties will be submitted to the project leading group for consideration.

\section{Response from the receiver}

The receiver project manager will confirm receipt within three working days of receiving the RFC and explain the time required to analyze the RFC and make the corresponding Engineering Change Recommendation (ENGINEERING CHANGE PROPOSAL, hereinafter referred to as ECP). The receiver party may charge the RFC analysis report and the ECP and inform the customer of the charging standard in writing. The receiver party will analyze the RFC and make the corresponding ECP within 30 days or within the agreed time after the customer agrees to the charging standard. 

The ECP will explain the following aspects of the impact of changes proposed in the RFC on the entire project.

\begin{itemize}

\item \textbf{Basic changes} -- the shape and performance of the car

\item \textbf{Flying design} -- performance, stability, reliability changes

\item \textbf{Test project} -- modification of test plan, test and retest

\item \textbf{Flying performance} -- confirm that the impact of changes in the flying hardware on the performance of the flying car has increased and whether it is necessary to modify other parts

\item \textbf{Training} -- training programs, course preparation and teaching materials

\item \textbf{Repair} -- Instructions for changing the repair and maintenance of the flying car

\item \textbf{Personnel needs} -- confirm whether additional personnel are needed to assist with changes to the speeding project

\item \textbf{Progress} -- progress of the project, speed of delivery of the speeding vehicle and expiration date of the agreement

\item \textbf{Possible cost}

\end{itemize}

\section{Approval of the applicant}

The applicant's project manager is required to confirm the ECP in writing. Any dispute that cannot be resolved by the project manager of both parties will be submitted to the project leadership team for review.

After the applicant's project manager confirms, if the modification involves the project contract or cost, it must be approved by the project leading group.

The approved ECP will be listed as an agreement in the Statement of Work in the form of an “Engineering Change Proposal”, replacing any priora sudden agreement.

\section{Change implementation}

Both parties will re-adjust the project plan based on the approved ECP and assign tasks.

Both parties will perform their respective responsibilities in accordance with the new project plan.

\section{Flow of variation procedure}

\begin{itemize}
\item The customer or the Recipient submits the RFC in writing;

\item Submit the RFC to the other party (or the project leadership team) for a technical feasibility assessment;

\item The customer gives the ECP  preparation time and required fees in writing;

\item The project manager appoints a review panel to discuss the time and cost of the customer  and whether to approve the RFC;

\item The customer makes the ECP and confirms the required fees and progress;

\item The two parties (or project leadership groups) discuss the ECP and propose implementation recommendations;

\item The applicant approves the ECP;

\item The project leadership team approves the modification of the contract (if needed);
Implement ECP.
\end{itemize}



