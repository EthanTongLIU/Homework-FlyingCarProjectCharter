%!TEX root = ../document.tex
\chapter{Project Document Management}

\section{Importance of project document management}

The implementation of the flying car project manufacturing management system is a complex system. To ensure the ultimate success of the project, it must be strict control is carried out at every stage of the purpose. The project's documentation reflects the project's work process and results and is the basis of project control. According to it, it is also a key carrier of “knowledge transfer”, so it is necessary to fully document the entire process of the project.

This document specifies the documents that need to be written during the project process, including project management documents, project technical documents and projects function documentation, etc. In addition, this document also describes the specific requirements for documentation, and the project team members are making these documents. It must be carried out in accordance with these requirements and must be signed by the corresponding responsible person.

\section{Project document system}

The corresponding documents need to be written at different stages of project implementation. The table below shows which documents are needed at which stages of the project, and the corresponding file format, encoding rules and required date of completion.

\renewcommand\arraystretch{1.2}
\begin{table}[H]
\centering
\footnotesize
\begin{tabular}[b]{|p{5cm}<{\raggedright}|p{5cm}<{\raggedright}|p{2cm}<{\raggedright}|p{2cm}<{\raggedright}|}
\hline
File name &	Project stage &	File time &	signature \\
\hline
Project implementation and work plan & 	The beginning of the project & &	\\
\hline	
User demand report & 	The beginning of the project &  &		\\
\hline
Evaluation report	 & All the stage & &		\\
\hline
Design Report & 	The beginning of the project & &	\\
\hline	
Design evaluation report	 & After design test &  &		\\
\hline
Material purchase application & 	Middle stage &  &		\\
\hline
Test result report and assessment	 & Middle stage &  &	\\
\hline	
User Training Program & 	Middle stage &  &		\\
\hline
User Manual	 & Middle stage &  &		\\
\hline
User training materials & 	Middle stage &  &		\\
\hline
Flying car production plan & 	Last stage &  &		\\
\hline
Maintenance Documentation & 	Last stage	 &  &	\\
\hline
Reconciliation Confirmation Report & 	Last stage &  &		\\
\hline
Product Quality Check Report	 & Last stage	 &  &	\\
\hline
\end{tabular}
\end{table}

The main contents and purpose of the report listed in the above table are explained below:

\begin{itemize}

\item \textbf{Project implementation and work plan:}

At the beginning of the project, the overall time plan, key checkpoints, division of duties and other things make it clear. In addition, in the specific implementation process, there must be a specific work plan, which is generally formulated and checked on a weekly basis.

\item \textbf{User demand report:}

Before the start of the project, collect all the user's needs for flying car products, and evaluate the rationality and operability of the demand.

\item \textbf{Evaluation report:}

It is the main work result of the system evaluation stage. It summarizes all current business processes, as well as all current business processes and current system inputs (forms, etc.) and outputs (reports, etc.). It should also include a series of function check tables that maps system functions to current business processes at a high level and finds differences from current processes/systems. Besides the report should also include key system interface requirements and data migration strategies.

\item \textbf{Design Report:}

Based on the design of the system, summarize what changes need to be made to the current business process. In based on the identified process, it should be summarized what configuration of the system is to meet the requirements of the process. For the system to be performed customized development includes reports, and it is up to the functional staff to develop requirements for the effects that development should achieve from a functional perspective.

\item \textbf{Design evaluation report:}

Conduct comprehensive experimental and simulation tests on flying car design, evaluate the test result data, analyze the rationality of the design change, and propose modifications to the design plan.

\item \textbf{Material purchase application:}

Determine the design drawings of all components and the type, quantity and price of the materials used, and apply for ensuring the rational use of the materials applied and minimizing production costs.

\item \textbf{Test result report and assessment:}

Collect and analyze the test flight results of the flying car model machine, propose revisions to the problems that arise in the flying car model, and summarize the impacts of subsequent problems, including human resources, material additions, extension of completion time and so on.

\item \textbf{User test cases and results:}

Flying car project team members should write system integration test cases according to their determined business processes. Make sure these cases include all of their business processes. Follow the impact of these business cases on the current process and the impact on other processes and steps should also be included in the case. When the user conducts the test, the actual test should be recorded. The results are compared with the expected results.

\item \textbf{User Training Program:}

Develop a training plan prior to the start of the training, and arrange the training process, course, and participants. 

\item \textbf{User Manual:}

User's guidance and reference manual for operating the system. It should include all that is determined by Ethane Autoplane Co., Ltd. Business processes and the functions of the system are organized in a way that is a business process. After the new user has received training in system use, according to the policy of the manual and business process, the operation of the system should be completed. The manual will also serve as an end user part of the materials used in the system training.

\item \textbf{User training materials:}

The materials used for end-user training are combined with the user manual to train end users.

\item \textbf{Maintenance Documentation:}

A maintenance manual for the flying car administrators. It should be written in conjunction with policy. It should include a description of the flying car architecture and the software and hardware platforms used; distribution (production, development, testing, etc.); system startup, shutdown, backup, performance monitoring, and common maintenance instructions for use, etc.

\item \textbf{Reconciliation Confirmation Report:}

Verifies each month's salary calculation results during the support period. 

\item \textbf{Product Quality Check Report:}

Summarize the status of the system up to date and analyze the main problems and solutions in the past. Advice on further improvements and improvements in system use.

\end{itemize}

All project documentation is written in English only. The signature of the document is, in principle, completed by the manufacturing operations leader. For the key project phase summary reports such as the system online report, it is necessary to report and discuss at the project meeting, and finally the leadership sign is confirmed. At the same time, due to the tight schedule of the project, to ensure that the project can proceed according to the plan and ultimately
On the planned date, the signing of the project document should be completed within 5 working days of the submission of the document. If more than 5 jobs it is still not signed, it will be deemed to have been signed and confirmed, and the work of the project team will be carried out in accordance with these documents.

\section{Project document management environment}

As part of project management best practices, project documentation should be maintained within a centralized and controlled environment. This not only ensures the standardization of the project documentation, but more importantly, it provides a convenient platform for the knowledge sharing of project stakeholders. The documentation for this project is maintained in the Flying Car Project Document Management Server. The specific address is: www.XXX.com.