%!TEX root = ../document.tex
\chapter{Introduction of Projrct Charter}

\section{Overview}

As a pioneer in the automobile-aviation industry, in order to promote the development of this industry, make the airplane commonplace, perpetuate the company and strengthen the public image and its position in the current markets, Ethane Autoplane Co., Ltd. decided to launch this project of Flying Car. This project charter, as a multi-agreement document, will include definitions of project objectives, development of implementation strategies, validation of project components and responsibilities and planning of project work. In order to ensure that the project implementation achieves the desired goals, the signing of the document will give the company implementation team authority and responsibility to carry out the work.

\section{Background}

Flying car is a personal vehicle that is capable of door-to-door aerial transport conceptually without the need of special take-off and landing as is required in an aircraft while also providing the comfort of a roadable car. Flying car would be used for shorter distances, at higher frequency, and at lower speeds and lower altitudes than conventional passenger aircraft. Though the concept seems to be a workable one but bringing it on the commercial front has been very challenging task and a surprising number of companies have been working to bring up an acceptable model. Many prototypes have been built since the first years of the twentieth century using a variety of flight technologies, but no flying car has yet reached production status.

The mechanical challenges of flying car are so strict that every opportunity must be taken to keep a minimum weight but at the same time a typical lightweight airframe is easily damaged. On the other hand, a road vehicle must be able to withstand significant impact loads from casual incidents as well as low-speed and high-speed impacts, and the high strength this demands can add considerable weight. Thus, a practical flying car must be both strong enough to pass road safety standards and light enough to fly. 

Since the flying car would be used at lower speeds and lower altitudes than conventional passenger aircraft, and the optimal fuel efficiency for airplanes is obtained at high altitudes and high subsonic speeds, the flying car's energy efficiency would be low compared to a conventional aircraft. Similarly, the flying car's road performance would be compromised by the requirements of flight, so it would be less economical than a conventional motor car as well. Our goal is to strike a balance between the two and maximize the benefits.

\section{Objective}

A mass-produced affordable and practical airplane product would be made, marketed, sold, and maintained just like an automobile. And our flying car must be capable of safe, reliable and environmentally-friendly operation both on public roads and in the air. For widespread adoption it must also be able to fly without a qualified pilot at the controls and come at affordable purchase and running costs. We initially plan to complete all conceptual model designs by 2025 and achieve industrial production by 2030. And the pre-orders should be held before the industrial production undergoes. The further plan to launch the fully automated version is by the year 2035.

\subsection{Safety}

A major problem, which increases rapidly with wider adoption, is the risk of mid-air collisions. Another is the unscheduled or emergency landing of a flying car on an unprepared location beneath, including the possibility of accident debris. In mid-air collisions and mechanical failures, the aircraft could fall from the sky or go through an emergency landing, resulting in deaths and property damage. In addition, poor weather conditions, such as low air density, lightning storms and heavy rain, snow or fog could be challenging and affect the aircraft's aerodynamics.

\subsection{Reliability}

A basic flying car requires the person at the controls to be both a qualified road driver and aircraft pilot. This is impractical for the majority of people and so wider adoption will require computer systems to de-skill piloting. These include aircraft maneuvering, navigation and emergency procedures, all in potentially crowded airspace. Fly-by-wire computers can also make up for many deficiencies in flight dynamics, such as stability. A practical flying car may need to be a fully autonomous vehicle in which people are present only as passengers.

\subsection{Environmentally-friendly}

A flying car capable of widespread use must operate safely within a heavily populated urban environment. As people's awareness of environmental protection gradually increases, when the products are marked with “green labels”, it will further promote the company's development in the market. Therefore, the lift and propulsion systems must be quiet, and have safety shrouds around all moving parts such as rotors, and must not create excessive pollution. Green energy sources such as solar energy, wind energy and other clean energy sources should also be considered if conditions permit.

\section{Cost}

Since the Skyrunner’s first flight-capable personal vehicle is now available (but has not yet achieved mass production) on sale price of 119 thousand dollars. In order to attract customers, popularize our products and strengthen the company's position in the market, our initial pricing for the product is 100 thousand dollars. Therefore, maximizing the company's interests requires controlling costs at every stage.

The design and determination of the concept model is crucial, and it was carried out by a professional research team using numerical simulation technology, which cost within 10 million dollars. The adoption of new structural materials, the development of automated driving and the need for the propulsion system to be both small and powerful can at present only be met using advanced and expensive technologies. The cost of manufacture could therefore be as much as 100 million dollars. For product promotion and sales, the cost should be controlled within 1 million dollars.



