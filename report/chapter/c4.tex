%!TEX root = ../document.tex
\chapter{Project Plan}

\section{Project phases and key tasks}

\section{Timeline}

\section{Milestone}

% \renewcommand\arraystretch{1.3}
% \begin{table}[!htb]
% \centering
% % \caption{}
% % \label{tab:}
% \begin{tabular}[b]{|m{4cm}<{\centering}|m{6cm}<{\centering}|m{4cm}<{\centering}|}
% \hline
% Project phase & Milestone & Planned date \\
% \hline
%   &   &  \\
% \hline
%   &   &  \\
% \hline
%   &   &  \\
% \hline
%   &   &  \\
% \hline
%   &   &  \\
% \hline
% \end{tabular}
% \end{table}

\renewcommand\arraystretch{1.0}
\begin{longtable}{|m{4cm}<{\centering}|m{6cm}<{\centering}|m{4cm}<{\centering}|}
\caption{\textbf{Project milestones}}
\label{tab:milestones}\\
\toprule
Project phase & Milestone & Planned date \\
\midrule
\endfirsthead
\multicolumn{3}{c}{Continued \autoref{tab:milestones}}\\
\multicolumn{3}{c}{(following last page)}\\
\toprule
Project phase & Milestone & Planned date \\
\midrule
\endhead
\bottomrule
\multicolumn{3}{c}{(continued on next page)}
\endfoot
\bottomrule
\endlastfoot

\multirow{3}{*}{项目准备} &   &   \\
\cline{2-3}
                         &   &   \\
\cline{2-3}
                         &   &   \\
\cline{2-3}
                         &   &   \\
\midrule
   &   &   \\
\midrule
   &   &   \\
\midrule
   &   &   \\
\midrule
   &   &   \\
\midrule
   &   &   \\
\midrule
   &   &   \\
\midrule
   &   &   \\
\midrule
   &   &   \\
\midrule
   &   &   \\
\midrule
   &   &   \\
\midrule
   &   &   \\
\midrule
   &   &   \\
\midrule
   &   &   \\
\midrule
   &   &   \\
\midrule
   &   &   \\
\midrule
   &   &   \\
\midrule
   &   &   \\
\midrule
   &   &   \\

\end{longtable}

\section{Project plan execution and report}

The project manager is primarily responsible for monitoring the progress of the project. The project plan is the key document used to inform the progress and current status of the project. The project plan includes project phase, task, duration, resources, scheduled start and end dates, milestones, persons responsible, and deliverables. The Project plan will be maintained by \textbf{XXX} and will reflect the Project methodology planning phase.

Only in two cases can the entire baseline plan be redesigned. One is that the entire baseline plan should be updated whenever there is any scope change that fundamentally affects project progress. Similarly, when schedule or budget deviations are significant, benchmark plans need to be reworked to make performance reports meaningful again.

The execution and reporting of the project plan shall be carried out in accordance with the following procedures: each project team member shall be responsible for updating the actual progress according to the project plan and estimating how long it will take to complete the tasks assigned to him/her as part of the weekly project report meeting. The project management team meets every Friday to review project progress against the project plan. The review is based on a review of delays, focusing on identifying existing or potential task delays, assessing the impact on the project, and agreeing on action plans to be taken to mitigate the impact. Project managers highlight tasks that may be delayed (e.g., expected completion time is later than planned). The person in charge of the task should develop an action plan for potential delays to minimize the impact on other project work. The project team leader shall indicate the possible task delay in the problem section of the weekly status report, including a brief description of the problem, a brief description of the action plan to prevent the delay or the date of the new task, and the date shall indicate the impact on other tasks.