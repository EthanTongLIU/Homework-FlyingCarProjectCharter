%!TEX root = ../document.tex
\chapter{Project Plan}

This project will follow the system implementation methodology summarized by the company in the field of manufacturing system implementation. The project will follow this implementation methodology.

\section{Project phases and key tasks}

{\color{red}{WBS分解}}

According to the implementation methodology, the whole process of the project is devided into 4 following phases.

\begin{itemize}
	\item \textbf{Project preparation} --- \textit{Object definition}

	\item \textbf{Blue print design} --- \textit{Target decomposition}

	\item \textbf{Manufacturing} --- \textit{Aims achieved}

	\item \textbf{Acceptance of delivery} --- \textit{Customer value realization}
\end{itemize}

Each step of the four-step implementation method is divided into tasks in detail, and the specific work content, work time, work mode, person in charge and work results of each step are defined.

\section{Timeline}

According to the above implementation methodology, the specific implementation plan of this project is as follows, and the project work will be carried out according to this plan.

{\color{red}{甘特图}}

\section{Milestone}

A milestone is a point in time used to mark the events or major accomplishments of the project team, as well as to mark the progress of the project. The major milestones and associated timelines for the Flying Car project are as follows:

% \renewcommand\arraystretch{1.3}
% \begin{table}[!htb]
% \centering
% % \caption{}
% % \label{tab:}
% \begin{tabular}[b]{|m{4cm}<{\centering}|m{6cm}<{\centering}|m{4cm}<{\centering}|}
% \hline
% Project phase & Milestone & Planned date \\
% \hline
%   &   &  \\
% \hline
%   &   &  \\
% \hline
%   &   &  \\
% \hline
%   &   &  \\
% \hline
%   &   &  \\
% \hline
% \end{tabular}
% \end{table}

\renewcommand\arraystretch{1.0}
\begin{longtable}{|m{4cm}<{\centering}|m{6cm}<{\centering}|m{4cm}<{\centering}|}
\caption{\textbf{Project milestones}}
\label{tab:milestones}\\
\toprule
\textbf{Project phase} & \textbf{Milestone} & \textbf{Planned date} \\
\midrule
\endfirsthead
\multicolumn{3}{c}{Continued \autoref{tab:milestones}}\\
\multicolumn{3}{c}{(following last page)}\\
\toprule
\textbf{Project phase} & \textbf{Milestone} & \textbf{Planned date} \\
\midrule
\endhead
\bottomrule
\multicolumn{3}{c}{(continued on next page)}
\endfoot
\bottomrule
\endlastfoot

\multirow{3}{*}{\bfseries \footnotesize Project preparation} &   &   \\
\cline{2-3}
                         &   &   \\
\cline{2-3}
                         &   &   \\
\cline{2-3}
                         &   &   \\
\midrule
\multirow{3}{*}{\bfseries \footnotesize Blue print design} &   &   \\ 
\cline{2-3}
                         &   &   \\
\cline{2-3}
                         &   &   \\
\cline{2-3}
                         &   &   \\
\midrule
\multirow{3}{*}{\bfseries \footnotesize Manufacturing} &   &   \\
\cline{2-3}
                         &   &   \\
\cline{2-3}
                         &   &   \\
\cline{2-3}
                         &   &   \\
\midrule
\multirow{3}{*}{\bfseries \footnotesize Acceptance of delivery} &   &   \\ 
\cline{2-3}
                         &   &   \\
\cline{2-3}
                         &   &   \\
\cline{2-3}
                         &   &   \\

\end{longtable}

\section{Project plan execution and report}

The project manager is primarily responsible for monitoring the progress of the project. The project plan is the key document used to inform the progress and current status of the project. The project plan includes project phase, task, duration, resources, scheduled start and end dates, milestones, persons responsible, and deliverables. The Project plan will be maintained by \textbf{XXX} and will reflect the Project methodology planning phase.

Only in two cases can the entire baseline plan be redesigned. One is that the entire baseline plan should be updated whenever there is any scope change that fundamentally affects project progress. Similarly, when schedule or budget deviations are significant, benchmark plans need to be reworked to make performance reports meaningful again.

The execution and reporting of the project plan shall be carried out in accordance with the following procedures: each project team member shall be responsible for updating the actual progress according to the project plan and estimating how long it will take to complete the tasks assigned to him/her as part of the weekly project report meeting. The project management team meets every Friday to review project progress against the project plan. The review is based on a review of delays, focusing on identifying existing or potential task delays, assessing the impact on the project, and agreeing on action plans to be taken to mitigate the impact. Project managers highlight tasks that may be delayed (e.g., expected completion time is later than planned). The person in charge of the task should develop an action plan for potential delays to minimize the impact on other project work. The project team leader shall indicate the possible task delay in the problem section of the weekly status report, including a brief description of the problem, a brief description of the action plan to prevent the delay or the date of the new task, and the date shall indicate the impact on other tasks.